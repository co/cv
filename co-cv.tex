\documentclass[margin,line]{resume}%
 
\usepackage[utf8]{inputenc}
\usepackage[english,swedish]{babel}
\usepackage[T1]{fontenc}
\usepackage{graphicx,wrapfig}
\usepackage{MinionPro}
\usepackage{url}
\usepackage[colorlinks=true, a4paper=true, pdfstartview=FitV,
linkcolor=blue, citecolor=blue, urlcolor=blue]{hyperref}
\usepackage{enumitem}
\usepackage[protrusion=true,expansion=true,final]{microtype}
\pdfcompresslevel=9
\setlist[description]{font=\itshape}

\begin{document}
\SetTracking[spacing={800,0*,0*}]{encoding=T1}{210}
{\sc \Large \textls{CARL-OSCAR ERNEHOLM}}%
\begin{resume}
    \vspace{0.5cm}
    \begin{wrapfigure}{R}{0.5\textwidth}
         \vspace{-1cm}
        \begin{center}
        %\includegraphics[width=0.5\textwidth]{me}
        \includegraphics[width=0.5\textwidth]{me_smaller}
        \end{center}
         \vspace{-1cm}
    \end{wrapfigure}

\SetTracking[spacing={0*,0*,0*}]{encoding=T1}{100}
	\section{\textls{\mysidestyle personlig \\information}}\vspace{0.001mm}
	   %Carl-Oscar Erneholm\\
	Hangarvägen 5, 5 tr.  \\
	183 66 Täby  \\
	Sweden \\
	   %tel: +46 176-143 39\\
	tel: +46 76-880 98 71\\
	   %\href{mailto:coer@kth.se}{\texttt{coer@kth.se}} \\
	\href{mailto:co.erneholm@gmail.com}{\texttt{co.erneholm@gmail.com}} \\
	   %\href{http://www.coern.com}{\texttt{www.coern.com}}
	\href{http://www.github.com/co}{\texttt{www.github.com/co}}

	Jag föddes och växte upp i Stockholm, när jag var 14 år flyttade jag
	och min familj ut till Singö i norra Roslagen. Under studietiden i
	gymnasiet och på KTH bodde jag i Norrtälje, men numera bor jag i Täby dit
	jag flyttade efter examen.

	Jag har alltid varit intresserad av teknik och
	matematik. För mig var valet av utbildning ganska självklart, då
	jag tidigt bestämde mig för att jag ville bli civilingenjör.
	Förutom mitt stora intresse för programmering gillar jag även grafisk
	design, fotografering och evolutions teori.
	
	På fritiden gillar jag att jobba med olika programmeringsprojekt, jag håller
	bland annat på med spelutveckling på lediga stunder.

	\section{\textls{\mysidestyle utbildning}}\vspace{0.001mm}

	\textbf{Civilingenjör i Datateknik, Kungliga Tekniska Högskolan}
	(2010-2012)\\
	Med inriktning på datalogi och programsystemteknik.
	Utbildningen har gett mig en inblick i en lång rad programmeringsspråk och
	tekniker för mjukvaruutveckling. Titel på examensarbete: \textit{``Automatic Summary Generation of Swedish
	Documents''}

	\textbf{Kandidat i Datateknik, Kungliga Tekniska Högskolan}
	(2007-2011)\\
	Titel på examensarbete: \textit{``Simulation of the Flocking Behavior of
	Birds with the Boids Algorithm''}

	\textbf{Studentexamen, Teknikprogrammet i Rodengymnasiet, Norrtälje} (2004-2007)\\
	På teknikprogrammet riktade jag in mig på IT och mattematik,
	för att få behörighet att söka civilingenjörsutbildningar.

\newpage
	\section{\textls{\mysidestyle erfarenhet}}\vspace{0.001mm}

		\textbf{Tobii Technology - Mjukvaruutvecklare, Stockholm } (2012 - \dots)\\
		Jag arbetar för närvarande som utvecklare av Tobiis mjukvara för analys av
		eyetracking data. Den största tiden har jag utvecklat mjukvarusviten
		till Tobiis eyetrackingglasögon.
		
		Under glasögonprojektet har jag haft möjlighet att arbeta med en mängd olika
		teknologier. Till största del har jag använt C\# .NET och WPF för
		analysmjukvaran. Men jag har också arbetat mer hårdvarunära på firmwaresidan
		av glasögonen, där utvecklade jag i C och Python, dessutom arbetade jag
		en hel del med Linux/Unix configurering.

		\textbf{Findwise - Civ.ing examensarbete, Stockholm } (2012 Jan-Jun)\\
		I civilingenjörsexamensarbeten ska man arbeta tillsammans med ett företag, för att
		förutom att skriva en examensrapport, implementera något praktiskt.
		Hos Findwise implementerade jag bl.a. två multidokumentsammanfattare för
		svenska texter till mitt examensarbete. Findwise är ett konsultföretag som
		riktat in sig på utveckla söklösningar för företag.

		\textbf{Tobii Technology – Quality Assurance, Stockholm} (2011 Sep-Dec)\\
		Jag regressionstestade en ny release av Tobii Studio. Tobii Studio är deras
		programvara miljö som används för att utföra statistiska tester med deras
		eye-trackers. Jag slutade för att kunna arbeta heltid med mitt Civ.ing
		examensarbete till våren.

		\textbf{SingöAffären - Butiksbiträde, Singö} (2010 Jun-Jul, 2011 Jun-Jul)\\
		Nyöppnad livsmedelsbutik ute i Roslagen. Jag fick ansvara för att ta emot varor, ta
		hand om kunder, kassatjänst etc.

	\section{\textls{\mysidestyle förtroende-\\uppdrag}}\vspace{0.001mm}

		\textbf{KTH Datasektionens mottagning av nyantagna studenter} (2010
		Sep-Oct, 2011 Sep-Oct)\\
		I två år var jag med och anordnade Datasektionen på KTH:s mottagning. Den
		rollen innefattade bland annat att: ansvara för och vara med att anordna
		olika evenemang samt umgås med, stötta, hjälpa, lära och få de nyantagna att trivas
		med sin nya utbildning.

	   \section{\textls{\mysidestyle programmerings-\\språk}}\vspace{0.001mm}

		\textbf{C\#}\hspace{0.5cm}
		Mitt bästa programmeringsspråk, större delen av min tid som utvecklare
		har jag jobbat med C\# .NET, jag känner mig mycket trygg i detta språk.

		\textbf{Python}\hspace{0.5cm}
		Python är mitt favorit skriptspråk, och ett av de språk jag har mest
		erfarenhet av. De flesta av mina hobbyprojekt brukar bli skrivna i Python.

		\textbf{Java}\hspace{0.5cm}
		Ett av mina bästa programmeringsspråk, det första jag lärde mig
		ordentligt. Mina erfarenhet kommer framförallt från min tid på KTH då
		jag använde det på de flesta programmerings kurserna.

		\textbf{C}\hspace{0.5cm}
		Jag känner mig trygg i att använda C, men jag har inte använt det lika mycket som
		C\# och Python. Jag har använt det dels i arbetslivet samt i en handfull kurser på KTH.

		\textbf{C++}\hspace{0.5cm}
		På KTH läste jag en kurs i C++, vilket innefattade, ett par mindre
		uppgifter och ett lite större projekt. Jag behärskar språket C++ men har
		begränsad erfarenhet.

		\textbf{Ruby}\hspace{0.5cm}
		Ruby är ett intressant skriptspråk och det första skriptspråket jag lärde
		mig. Jag har använt det en del, men jag är nog ganska rostig.

\newpage
	   \section{\textls{\mysidestyle tekniskt\\kunnande}}\vspace{0.001mm}

		\textbf{Unix}\hspace{0.5cm}
		Jag har mycket god erfarenhet av Unix och jag har använt det regelbundet
		sen 2007. Jag hanterar en terminal med lätthet och är van
		vid att koppla ihop diverse Unix-program när det behövs.

		\textbf{Windows}\hspace{0.5cm}
		Windows är det operativ system jag har använt större delen av mitt liv.
		Jag är alltså mycket van vid att arbeta i Windows miljö.

		\textbf{Git}\hspace{0.5cm}
		Jag har lång erfarenhet av git, jag började använda det redan på KTH
		och har använt det regelbundet sen dess, både i arbetslivet och på fritiden.

		\textbf{\LaTeX}\hspace{0.5cm}
		Latex är ett kraftfullt typsättningssystem för att skapa typografiskt
		korrekta dokument. Jag har goda erfarenheter med detta system och på KTH
		använde jag det för att skriva flera av mina uppsatser och båda mina
		examensarbeten. Även detta dokument är skrivet i latex.

		\textbf{HTML, CSS, PHP}\hspace{0.5cm}
		Jag har skrivit en handfull hemsidor i HTML och CSS, jag har inga
		problem att skriva enklare hemsidor. PHP har jag använt endast
		några enstaka gånger.

%TODO Language Skills.

\end{resume}
\end{document}
