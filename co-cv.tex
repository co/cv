\documentclass[margin,line]{resume}%
 
\usepackage[utf8]{inputenc}
\usepackage[english,swedish]{babel}
\usepackage[T1]{fontenc}
\usepackage{graphicx,wrapfig}
\usepackage{MinionPro}
\usepackage{url}
\usepackage[colorlinks=true, a4paper=true, pdfstartview=FitV,
linkcolor=blue, citecolor=blue, urlcolor=blue]{hyperref}
\usepackage{enumitem}
\usepackage[protrusion=true,expansion=true,final]{microtype}
\pdfcompresslevel=9
\setlist[description]{font=\itshape}

\begin{document}
\SetTracking[spacing={800,0*,0*}]{encoding=T1}{210}
{\sc \Large \textls{CARL-OSCAR ERNEHOLM}}%
\begin{resume}
    \vspace{0.5cm}
    \begin{wrapfigure}{R}{0.5\textwidth}
         \vspace{-1cm}
        \begin{center}
        %\includegraphics[width=0.5\textwidth]{me}
        \includegraphics[width=0.5\textwidth]{me_smaller}
        \end{center}
         \vspace{-1cm}
    \end{wrapfigure}

\SetTracking[spacing={0*,0*,0*}]{encoding=T1}{100}
	   \section{\textls{\mysidestyle personlig \\information}}\vspace{0.001mm}
	   Carl-Oscar Erneholm\\
	   Bergsgatan 14 F 3 tr.  \\
	   761 42 Norrtälje  \\
	   Sweden \\
	   %tel: +46 176-143 39\\
	   tel: +46 76-880 98 71\\
	   \href{mailto:coer@kth.se}{\texttt{coer@kth.se}} \\
	   %\href{http://www.coern.com}{\texttt{www.coern.com}}

	Jag föddes och växte upp i Stockholm, när jag var 14 år flyttade jag
	och min familj ut till Singö i norra Roslagen. Sedan dess har jag
	flyttat till Norrtälje där jag även gick i gymnasiet. För närvarande
	bor jag kvar i Norrtälje med min studerande lillasyster, men jag planerar att
	flytta till tillbaka till Stockholm när jag tagit civilingenjörs examen.

	Enda sedan jag var liten har jag varit intresserad av teknik och
	matematik. För mig var valet av utbildning ganska självklart, då
	jag tidigt bestämde mig för att jag ville bli civilingenjör.
	Förutom mitt stora intresse för programmering gillar jag även grafisk
	design, fotografering och evolutions teori. På fritiden gillar jag
	att ta promenader, resa och umgås med vänner.

	\section{\textls{\mysidestyle utbildning}}\vspace{0.001mm}

	\textbf{Civilingenjör i Datateknik, Kungliga Tekniska Högskolan}
	(2010-2012).\\
	Med inriktning på datalogi och programsystemteknik.
	Utbildningen har gett mig en inblick i en lång rad programmeringsspråk och
	tekniker för mjukvaruutveckling. Titel på examensarbete: \textit{``Automatic Summary Generation of Swedish
	Documents''}

	\textbf{Kandidat i Datateknik, Kungliga Tekniska Högskolan}
	(2007-2011).\\
	Titel på examensarbete: \textit{``Simulation of the Flocking Behavior of
	Birds with the Boids Algorithm''}

	\textbf{Studentexamen, Teknikprogrammet i Rodengymnasiet, Norrtälje} (2004-2007).\\
	På teknikprogrammet riktade jag in mig på IT och mattematik,
	för att få behörighet att söka civilingenjörsutbildningar.

	\section{\textls{\mysidestyle erfarenhet}}\vspace{0.001mm}

		\textbf{Findwise - Civ.ing examensarbete, Stockholm } (2012 Jan-Jun).\\
		I civilingenjörsexamensarbeten ska man jobba tillsammans med ett företag, för att
		förutom att skriva en examensrapport, implementera något praktiskt.
		Hos Findwise implementerade jag bl.a. två multidokumentsammanfattare för
		svenska texter till mitt examensarbete. Findwise är ett konsultföretag som
		riktat in sig på utveckla söklösningar för företag.

\newpage
		\textbf{Tobii Technology – Quality Assurance, Stockholm} (2011 Sep-Dec).\\
		Jag regressionstestade en ny release av Tobii Studio. Tobii Studio är deras
		programvara miljö som används för att utföra statistiska tester med deras
		eye-trackers. Jag slutade för att kunna jobba heltid med mitt Civ.ing
		examensarbete till våren.

		\textbf{SingöAffären - Butiksbiträde, Singö} (2010 Jun-Jul, 2011 Jun-Jul).\\
		Nyöppnad livsmedelsbutik ute i Roslagen. Jag fick ansvara för att ta emot varor, ta
		hand om kunder, kassatjänst etc.

	\section{\textls{\mysidestyle förtroende-\\uppdrag}}\vspace{0.001mm}

		\textbf{KTH Datasektionens mottagning av nyantagna studenter} (2010
		Sep-Oct, 2011 Sep-Oct).\\
		I två år var jag med och anordnade Datasektionen på KTH:s mottagning. Den
		rollen innefattade bland annat att: ansvara för och vara med att anordna
		olika evenemang samt umgås med, stötta, hjälpa, lära och få de nyantagna att trivas
		med sin nya utbildning.

	   \section{\textls{\mysidestyle programmerings-\\språk}}\vspace{0.001mm}

		\textbf{Java}\hspace{0.5cm}
		Mitt bästa programmeringsspråk, det första jag lärde mig ordentligt och
		även det jag använt absolut mest. Jag har erfarenhet av java från hela min
		tid på KTH.

		\textbf{C\#}\hspace{0.5cm}
		Mitt näst bästa programmeringsspråk, jag har kodat ett projekt i C\#,
		jag känner mig mycket trygg i detta språk,
		mestadels för att det är mycket likt java.

		\textbf{C}\hspace{0.5cm}
		Jag känner mig rätt trygg i C, men jag har inte använt det lika mycket som
		java. Jag har använt det i en handfull kurser på KTH.

		\textbf{C++}\hspace{0.5cm}
		På KTH läste jag en kurs i C++, vilket innefattade, ett par mindre
		uppgifter och ett lite större projekt. Kan skriva bra kod hjälpligt.

		\textbf{Ruby}\hspace{0.5cm}
		Ruby är förmodligen mitt favorit-skriptspråk och jag har använt det till
		och från de senaste åren. Jag känner mig trygg i att arbeta med Ruby.

		\textbf{Python}\hspace{0.5cm}
		Python är det senaste skriptspråk jag plockat upp, jag kan det nästan lika
		bra som Ruby.

	   \section{\textls{\mysidestyle tekniskt\\kunnande}}\vspace{0.001mm}

		\textbf{Unix}\hspace{0.5cm}
		Jag har goda Unix-vanor och fem års erfarenhet av av diverse GNU/Linux
		distributioner. Jag hanterar en terminal med lätthet och är van
		vid att koppla ihop diverse Unix-program när det behövs.

		\textbf{Windows}\hspace{0.5cm}
		Windows är det operativ system jag har använt större delen av mitt liv.
		Jag är alltså mycket van vid att jobba i Windows miljö.

		\textbf{\LaTeX}\hspace{0.5cm}
		Latex är ett kraftfullt typsättningssystem för att skapa typografiskt
		korrekta dokument. Jag har goda erfarenheter med detta system, på KTH
		använde jag det för att skriva flera av mina uppsatser och båda mina
		examensarbeten. Även detta dokument är skrivet i latex.

		\textbf{HTML, CSS, PHP}\hspace{0.5cm}
		Jag har skrivit en handfull hemsidor i HTMl och CSS, jag har inga
		problem att skriva enklare hemsidor. PHP, har jag använt i endast
		några enstaka gånger, jag kan använda det hjälpligt.

		\textbf{MySQL}\hspace{0.5cm}
		På KTH gick jag en grundläggande Databas-kurs där vi använde MySQL,
		jag kan använda MySQL hjälpligt.

%TODO Language Skills.
%TODO git.

\end{resume}
\end{document}
